%=================================================================
\section{Introduction}\label{sec-intro}


%Test citation~\cite{BL12J01}. 
%\begin{JournalOnly}
%and~\citep{BJL11J01} or~\citet{BJL11J01}.
%\end{JournalOnly}

%This is for~\cref{tbl:overall-experiments}, 
%\todo[fancyline]{Testing.}
%and this is for~\cref{sec-conclusions}.
%\todo[noline]{A note with no line back to the text.}%
%\gangli{This is comment from Gang.}
%\qwu{Response from QW}

%Number:
%\num{123}.
%\numlist{10;30;50;70},
%\numrange{10}{30},
%\SIlist{10;30;45}{\metre},
%and
%\SI{10}{\percent}

%\missingfigure[figcolor=white]{Testing figcolor}
\subsection{Problem Description}
\

Many social programs today have difficulty ensuring that adequate assistance is provided to the right people. This is especially tricky when a project focuses on the poorest. The poorest people in the world are often unable to provide the necessary income and expenditure records to prove that they are eligible. In Latin America, we need to qualify for revenue. It is referred to as agent means test (or PMT). Through PMT, agents use a model, which can analyze the needs of their families by considering the observable family properties, such as the materials of walls and ceilings, or the assets found in their homes. Although this is an improvement, accuracy remains a problem as the population of the region grows and poverty decreases.
%\subsection{Data List}
\

%The data contains date of the item sold at a store ,the name of the store, the name of the item,
%and the sales of every store on a particular day.
%\begin{description}
  %\item [date] - Date of the sale data. There are no holiday effects or store closures.
  %\item [store] - Store ID
  %\item [item] - Item ID
  %\item [sales] - Number of items sold at a particular store on a particular date.
%\end{description}

%\subsection{Problem Analysis}
\

%This is a question that asks us to predict sales for three months
 %based on sales that have been available for three years. And We can
 %analyze that this is a regression problem.

\section{Problem Analysis} \label{sec-data_exploration}

\subsection{Data Information}
\

The data is divided into two files: training set train.csv and test set test.csv.

Each row represents the data of a family member (a family can be composed of multiple members, only the head of the household is predicted)

Training set: train.csv, 143 columns, ID, target and 141 features

Test set test.csv, including 142 columns, excluding target

The head of household represents a family, and only the head of household is graded

Key fields:

ID: an identification of each sample data

Idhogar: the unique identification of each family, with the same identification sample belonging to a family

Parentesco1: indicates whether the member is the head of the household. Equal to 1 means the member is the head of the household

Target: poverty of families

1 = extreme poverty

2 = moderate poverty

3 = vulnerable households

4 = non vulnerable households

1 is the most extreme poverty, 2 is the poor, 3 is vulnerable, 4 is not vulnerable

%\begin{table}[htbp]  \centering
	%\caption{Data Information}
	%\label{tbl:data information}
	%\begin{tabular}{ccccccc}
		%\hline
		% after \\: \hline or \cline{col1-col2} \cline{col3-col4} ...
		%& date & store & item & sales\\
		%\hline
		%0 & 2013-01-01 & 1 & 1 & 13 \\
	 % 1 & 2013-01-02 & 1 & 1 & 11 \\
		%2 & 2013-01-03 & 1 & 1 & 14 \\
		%3 & 2013-01-04 & 1 & 1 & 13 \\
	  %4 & 2013-01-05 & 1 & 1 & 10 \\
		%\hline 
		%\bottomrule
	%\end{tabular}
%\end{table}

\subsection{Data Analysis}

\subsubsection{Addressing Wrong Target}
\

The normal situation is: if the target of a family is extreme, that is, extreme poverty, that is, the target of the head of household is extreme, then the target of all members of the family should be the same extreme. However, there are some exceptions, the members of the same family have multiple different targets, which may be caused by human or other factors.
First, find families with different family member labels. Then, we can correct all the label differences by assigning the label of the head of household to everyone in the same household.
For a family without a head of household, we don't need to worry. If a family has no head of household, then there is no real label.

\subsubsection{Missing Value Handling}
\

Missing value processing I take the following approach. First look at the existing data, with the most cases as a reference. For example, the most common number of tablets in a family is 1, so we fill in the missing places with 1.

\subsection{Data Visualization}
\

Data visualization is a scientific and technological research on the visual representation of data. Among them, the visual representation of data is defined as information extracted in some summary form, including various attributes and variables of corresponding information units. Data visualization mainly aims to convey and communicate information clearly and effectively by means of graphical means.

\subsubsection{Data Glance}
\

Here, the data list shows the first five rows of data, that is, the situation of five families. The attributes include ID, room number, target tag, etc.

%By using the matplotlib to plot the photoes which describe the
%sale pattern. For example, The distribution of sales volume ,total 
%sales of the stores, total sales of the items, store’s performance 
%over the time, all items’ performance over the time ect.

%From the figures 1 we can know that 2nd store is the topper and
%7th store is the least revenue generating one; From figure 2 we can
%know that more and more sales volume is belong to [0,50]; From figure 
%3 we can know that 2nd store is the topper of the all stores; From
%figure 4 we can know the total sales of all items; From figure 5 we 
%can know every stores sales changing overtime; From figure 6 we can
%know every item sales changing overtime; From figure 7 and 8 we can 
%know individual pattern of item’s sale and store’s sale.



\subsubsection{Histogram}
\

Histogram, also known as mass distribution map, is a kind of statistical report map, which is represented by a series of vertical stripes or line segments with different heights Generally, the horizontal axis is used to represent the data type, the vertical axis is used to represent the distribution, and the histogram is the accurate graphic representation of the numerical data distribution.From this histogram, it can be seen that this is obviously a category imbalance, belonging to the fourth category, that is, non vulnerable families have the most people. From this histogram, it can be seen that this is obviously a category imbalance, belonging to the fourth category, that is, non vulnerable families have the most people.


\subsubsection{Box Diagram}
\

Box chart, also known as box chart, box chart or box line chart, is a statistical chart used to display a group of data dispersion information. It is mainly used to reflect the distribution characteristics of the original data, and also to compare the distribution characteristics of multiple groups of data. It can be found from the box chart that the average education level of the extremely poor families with poverty level of 1: extreme is the lowest, both male and female heads of household. With the increase of education level, the poverty level has declined, and it can be found that the average education level of edu and povert is inversely proportional. No matter which level of poor families, the average education level of female heads of household is slightly higher than that of male heads of household.


\begin{figure}[htbp]
	\centering
	
	%\graphicspath{{figures/}{mine/}}
	\includegraphics[scale=1.0]{E:/pictures/f3.eps}
	\caption{Data characteristics table}\label{fig:f3.eps}
\end{figure}

\begin{figure}[htbp]
	\centering
	
	%\graphicspath{{figures/}{mine/}}
	\includegraphics[scale=0.5]{E:/pictures/f2.eps}% 新的文件路径为 logos/f2.eps
	\caption{Poverty level distribution}\label{fig:f2.eps}
\end{figure}

\begin{figure}[htbp]
	\centering
	
	%\graphicspath{{figures/}{mine/}}
	\includegraphics[scale=1.0]{E:/pictures/f1.eps}
	\caption{Relation diagram}\label{fig:f1.eps}
\end{figure}
%\begin{figure}[htbp]
	%\centering
	
	%\includegraphics[scale=0.8]{E:/tulip-flip/templatex-master/powerdot-tuliplab/logos/0004.eps}
	%\caption{ sales volume’s distribution}\label{fig:002.eps}
%\end{figure}

%\begin{figure}[htbp]
	%\centering
	
	%\includegraphics[scale=0.6]{E:/tulip-flip/templatex-master/powerdot-tuliplab/logos/0005.eps}
	%\caption{ The total sales of all stores}\label{fig:003.eps}
%\end{figure}

%\begin{figure}[htbp]
	%\centering
	
	%\includegraphics[scale=0.3]{E:/tulip-flip/templatex-master/powerdot-tuliplab/logos/0006.eps}
	%\caption{ The total sales of all items}\label{fig:004.eps}
%\end{figure}

%\begin{figure}[htbp]
	%\centering
	
	%\includegraphics[scale=0.3]{E:/tulip-flip/templatex-master/powerdot-tuliplab/logos/0007.eps}
	%\caption{  The performance of all stores}\label{fig:005.eps}
%\end{figure}

%\begin{figure}[htbp]
	%\centering
	
	%\includegraphics[scale=0.3]{E:/tulip-flip/templatex-master/powerdot-tuliplab/logos/0008.eps}
	%\caption{ The performance of all stores items}\label{fig:006.eps}
%\end{figure}

%\begin{figure}[htbp]
	%\centering
	
 %\includegraphics[scale=0.7]{E:/tulip-flip/templatex-master/powerdot-tuliplab/logos/0009.eps}
  %\includegraphics[scale=0.7]{E:/tulip-flip/templatex-master/powerdot-tuliplab/logos/0010.eps}
	%\caption{  The performance of the individual store and item}\label{fig:007.eps}
%\end{figure}


\ 

%Through the data visualization before, we can intuitively 
%recognize the changes in sales. However, to forecast sales for the 
%next three months, we need to extract some new features. From the
%previous figure we can see that the sales are related to the 
%characteristics of the year, month, season, etc., so we can add some 
%new features.
%The new features that are specifically added are as follows:
\ 
%\begin{description}
 % \item [dayofweek] -  Indicates that this day is the day of the week. Monday is \hspace*{1.8cm} indicated by 0, Tuesday is indicated by 1, and so on.
                       Sunday \hspace*{1.8cm} is indicated by 6.
  %\item [is\_weekend] - Determine if this day is a weekend. The weekend is indicated \hspace*{1.85cm} by 1 and the working day is represented by 0.
 % \item [day] - Judging this day is the first few days of this month.
  %\item [year] - Judging year.
  %\item [dayofyear] - Judging that day is the first few days of the year.
  %\item [weekofyear] - Judging that the day is the first few weeks of the year.
  %\item [sales\_mean\_lag\_90] - Calculate 90 days from the day before, and then start \hspace*{3.1cm} from this day, the average of the first 
  %seven days.
 % \item [sales\_std\_lag\_90] - Calculate 90 days from the day, and then start from this \hspace*{2.7cm} day, the standard deviation of the 
  %first seven days.
%\end{description}

\section{Methods}



\subsection{Random Forest}
\ 

Random forest is a classifier that uses multiple trees to train and predict samples. In machine learning, random forest is a classifier with multiple decision trees, and the output category is determined by the mode of the output category of individual tree. Random forest is an algorithm that integrates multiple trees through the idea of integrated learning. Its basic unit is decision tree, and its essence belongs to a big branch of machine learning integrated learning method.



%\begin{description}
  %\item  Faster training speed and higher efficiency.
  %\item  Lower memory usage.
  %\item  Better accuracy.
  %\item  Support of parallel and GPU learning.
  %\item  Capable of handling large-scale data
%\end{description}
\ 


\


\
%\[SMAPE=\frac{100\%}{n}\sum_{t=1}^n\frac{|F_t-A_t|}{(|A_t|+|F_t|)/2} \]
%\begin{ConferenceOnly}
%We have \SI{10}{\hertz},
%\si{\kilogram\metre\per\second},
%the range: \SIrange{10}{100}{\hertz}.
%$\nicefrac[]{1}{2}$.

%\missingfigure{Make a sketch of the structure of a trebuchet.}

%\end{ConferenceOnly}


%For~\cref{eq:test},
%as shown below:

%\begin{equation}\label{eq:test}
%a = b \times \sqrt{ab}
%\end{equation}

%\blindmathpaper

\section{Resuult}
\

From the predicted results, we can see that the number of people belonging to label four, that is, non vulnerable families is the most.
\begin{figure}[htbp]
	\centering
	
	%\graphicspath{{figures/}{mine/}}
	\includegraphics[scale=1.0]{E:/pictures/result.eps}
	\caption{Forecast result chart}\label{fig:result.eps}
\end{figure}

%\begin{figure}[htbp]
	%\centering
	
  %\includegraphics[scale=0.5]{E:/tulip-flip/templatex-master/powerdot-tuliplab/logos/012.eps}
	%\caption{  The value vs predict}\label{fig:008.eps}
%\end{figure}

%\begin{figure}[htbp]
	%\centering
	
  %\includegraphics[scale=0.7]{E:/tulip-flip/templatex-master/powerdot-tuliplab/logos/013.eps}
	%\caption{Top 10 importabce features }\label{fig:009.eps}
%\end{figure}

%\begin{figure}[htbp]
	%\centering
	
  %\includegraphics[scale=0.7]{E:/tulip-flip/templatex-master/powerdot-tuliplab/logos/014.eps}
	%\caption{ The test prediction }\label{fig:010.eps}
%\end{figure}
\ 

%Figure 9 is the first prediction based on the model. Figure 10 
%shows ten features with high feature importance. Figure 11 is a new
%prediction based on the model to get the results needed for this problem.

\section{Conclusion}
\begin{itemize}
 \item The data preprocessing part is very important, which is the basis to solve the later problems. In this part, data processing should be considered comprehensively.
 \item Random forest is a very flexible and practical method, which has excellent accuracy and can run effectively on large data sets.
 \item Drawing is also very important. You can clearly see the relationship between data.
\end{itemize}
%\section{Preliminaries} \label{sec-preliminaries}

%\blindtext

%\gliMarker  %TODO: GLi Here


%\section{Method} \label{sec-method}

%\blindtext
%\blindlist{itemize}[3]
%\blinditemize
%\blindenumerate

%\blindmathtrue
%\blindmathfalse
%\blinddescription

%\qwuMarker %TODO: QWu Here

%\section{Experiment and Analysis} \label{sec-experiment}


%\begin{table}  \centering
  %\caption{Precision Comparison on Event Detection Methods}
  %\label{tbl:overall-experiments}
  %\begin{tabular}{cccc}
%\toprule
    % after \\: \hline or \cline{col1-col2} \cline{col3-col4} ...
    %& OR Event Detection & AC Event Detection & TC Event Detection \\
%\midrule
    %precision & 0.83 & 0.69 & 0.46 \\
    %recall & 0.68 & 0.48 & 0.36 \\
    %F-score & 0.747 & 0.57 & 0.4 \\
%\bottomrule
%\end{tabular}
%\end{table}


%\section{Conclusions} \label{sec-conclusions}

%\blindtext

%\section*{Acknowledgment}

%\lipsum[1]


%The authors would like to thank \ldots

